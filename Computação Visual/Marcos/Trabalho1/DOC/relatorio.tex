\documentclass[a4paper, 12pt]{article}

\usepackage[brazil]{babel}
\usepackage[utf8]{inputenc}
\usepackage{amsmath}
\usepackage{indentfirst}
\usepackage{graphicx}
%\usepackage[colorinlistoftodos]{todonotes}
%\usepackage{listings}
\oddsidemargin=0in
\topmargin=0in
\headheight=0pt
\headsep=0pt
\parindent=0pt
\textwidth = 450pt
\textheight = 700pt

\begin{document}

	\begin{titlepage}

		\begin{center}

			\huge{Universidade Federal da Bahia}

			\vspace{100pt}

			\begin{figure}[!ht]
				\centering
				\includegraphics[width=3cm]{UFBA.png}
			\end{figure}
			        
			\vspace{75pt}

			\textbf{\LARGE{Relatório}}\\
			\large{Imagem Estéreo}

			\vspace{160pt}

			Marcos Antônio de Souza Silva\\
			Salvador - Bahia \date{today} \\

		\end{center}

	\end{titlepage}

	\newpage

	\pagenumbering{arabic}
	
	\section{Descrição do algoritmo implementado}

	O algoritmo implementado utiliza:

	\subsection{Longest Common Subsequence(LCS)}
	
	O LCS implementado foi baseado no algoritmo implementado como solução do problema do SPOJ - Parque Jurássico, http://br.spoj.com/problems/PARQUE/. 
	O algoritmo LCS encontra a maior subsequencia comum entre strings A e B. O algoritmo monta uma matriz, de inteiros, C de dimensão M+1xN+1, onde M é o tamanho da string A e N é o tamanho da string B. Onde o indice de cada linha de C representa o mesmo indice na palavra A e o indice de cada coluna representa o mesmo indice na palavra B. A primeira linha, linha 0, e a primeira coluna, coluna 0, da matriz C inicializados com zero. \\

	Percorre-se a matriz checando se os indices i e j correspondem a mesma letra nas duas palavras caso sejam iguais é acrescentado 1 a posição dos indices, C[i][j], recebe o valor C[i-1][j-1] + 1, para indicar que foi encontrada mais uma letra em comum. Ao final, a matriz C, conterá o tamanho da maior substring comum, até posição i e j das palavras.\\

	Adaptei o LCS imaginando que cada linha das duas imagens de entrada, a esquerda e a direta, fosse a string A e B, respectivamente. E assim, para cada linha i das imagens, 0 <= i < altura da imagem - 1, aplico o LCS, gerando uma matriz C, com dimensão WxW, onde W é o valor da largura da imagem e aplico o backtrack.

	\subsubsection{Backtrack}

	Após aplicar o LCS, a cada linha i, das imagens, utilizava o algoritmo backtrack para obter a menor sequência que continha as duas imagens como subsequências.

	\subsection{Cálcular a distância}

	Utiliza-se a raiz da distância absoluta, uma vez, que a mesma mostrou-se mais eficaz quando usada, do que os outras formas de cálculos da distância.

\end{document}